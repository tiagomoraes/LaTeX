\documentclass[a4paper]{article}

%% Language and font encodings
\usepackage[brazil]{babel}
\usepackage[utf8x]{inputenc}
\usepackage[T1]{fontenc}

%% Sets page size and margins
\usepackage[a4paper,top=3cm,bottom=2cm,left=3cm,right=3cm,marginparwidth=1.75cm]{geometry}

%% Useful packages
\usepackage{amsmath}
\usepackage{graphicx}
\usepackage[colorinlistoftodos]{todonotes}
\usepackage[colorlinks=true, allcolors=blue]{hyperref}

\title{IF672 - Algoritmos e Estruturas de Dados}
\author{Tiago Moraes}

\begin{document}
\maketitle

\section{Introdução}
\textbf{Algoritmos e Estrutura de Dados (IF672)} é uma das cadeiras obrigatórias componentes do curso de Ciência da Computação na Universidade Federal de Pernambuco. Correspondente ao segundo período do curso, a cadeira de Algoritmos, como é convencionalmente chamada, se insere na área de Computação Básica, trabalhando com o estudo das estruturas de dados e tendo como objetivo o desenvolvimento de sistemas mais eficientes, sem o comprometimento do design e da organização do código.

A disciplina abrange as áreas de Estrutura de Dados (ED), estudando algoritmos relacionados à otimização dos sistemas digitais, tais como algoritmos de ordenação mais rápida a exemplo do \textit{QuickSort} e do \textit{MergeSort}; Tabelas de dispersão (\textit{Hash tables}); Grafos e Árvores; Algoritmos gulosos para problemas de otimização em grafos; etc.. Tais conteúdos são considerados fundamentais para os currículos em Ciência da Cumputação, devendo ser estudados após o programador compreender os princípios de implementação e de design de programas claros, para que então, torne-os mais eficientes.

\section{Relevância}
O estudo de algoritmos é de extrema importância para a formação profissional de um desenvolvedor, pois é através deles que o programador é capaz de otimizar seu código, tornando o programa em desenvolvimento não só mais rápido, mas também mais simples e leve. Ainda que os computadores tornem-se cada vez mais rápidos e eficientes com o avançar da tecnologia, o estudo dos algoritmos de otimização é fundamental para o desenvolvimento de sistemas mais enxutos e simples em diversas áreas da computação.

É a partir do domínio de algoritmos que o aluno de graduação em Ciência da Computação consegue abstrair as lógicas de desenvolvimento, compreendendo mais claramente os códigos nos quais está trabalhando, bem como assimilando a lógica por trás dos comandos realizados através de uma linguagem de programação. Desssa forma, a disciplina de Algoritmos e Estrutura de Dados torna-se fundamental para o curso, sendo bastante valorizada pelo mercado, bem como considerada como base para o desenvolvimento de bons programas.

Ao analisar as Estruturas de Dados, isto é, os modos particulares de de armazenamento e organização de dados em um computador, pode-se utilizar as informações disponíveis de forma mais eficiente, facilitando sua busca e modificação. É através da organização e dos métodos de manipulação das EDs que consegue-se, por exemplo, reduzir o espeço ocupado na memória RAM ou o tempo de carregamento de um sistema.

\subsection{Pontos Positivos}
\begin{itemize}
	\item Desenvolvimento da capacidade de abstração do código, desvinculando-o de uma linguagem de programação específica;
	\item Desenvolvimento da capacidade de produção de sistemas otimizados e leves;
    \item Importante base teórica para o desenvolvimento de programas em diversas apliações;
    \item Conhecimento imprescindível para o mercado de trabalho na área de TI.
\end{itemize}

\subsection{Pontos Negativos}
\begin{itemize}
	\item Disciplina considerada pesada e com alta taxa de reprovação;
    \item Conteúdos pouco lineares e bastante amplos;
\end{itemize}

\section{Relaçao com outras disciplinas}

%Usar https://www.tablesgenerator.com/ para criar a tabela de comaração entre as disciplinas.
\begin{table}[h]
\centering
\caption{Cmparação}
\label{table:comparacao}
\begin{tabular}{lccccc}
Jogos                             & \multicolumn{1}{l}{Lançamento} & \multicolumn{1}{l}{Desenvolvedora} & \multicolumn{1}{l}{PS4} & \multicolumn{1}{l}{PC} & \multicolumn{1}{l}{XBOX} \\
\multicolumn{1}{l}{Battlefield 1} & 21/10/2016                     & Actvision                          & s                       & s                      & s                        \\
\multicolumn{1}{l}{COD WWII}      & 03/11/2017                     & EA DICE                            & s                       & s                      & s                        \\
Rainbow Six Siege                 & 01/12/2015                     & Ubisoft                            & s                       & s                      & s                       
\end{tabular}
\end{table}

\bibliographystyle{alpha}
\bibliography{references.bib}

\end{document}